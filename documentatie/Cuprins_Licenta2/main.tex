\documentclass[runningheads,a4paper,11pt,twoside]{report}

\usepackage{algorithmic}
\usepackage{algorithm}
\usepackage{array}
\usepackage{amsmath}
\usepackage{amsfonts}
\usepackage{amssymb}
\usepackage{amsthm}
\usepackage{caption}
\usepackage{comment}
\usepackage{epsfig}
\usepackage{fancyhdr}
\usepackage[T1]{fontenc}
\usepackage{geometry}
\usepackage{graphicx}
\usepackage[colorlinks]{hyperref}
\usepackage[utf8]{inputenc}
\usepackage[romanian]{babel}
\usepackage{multicol}
\usepackage{multirow}
\usepackage{rotating}
\usepackage{setspace}
\usepackage{subfigure}
\usepackage{url}
\usepackage{verbatim}
\usepackage{xcolor}
\usepackage{ragged2e}

\geometry{a4paper,top=3cm,left=2cm,right=2cm,bottom=3cm}

\pagestyle{fancy}
\fancyhf{}
\fancyhead[LE,RO]{Smart Agriculture}
\fancyhead[RE,LO]{Roberto Chiper}
\fancyfoot[RE,LO]{Licen\c{t}a 2024-2025}
\fancyfoot[LE,RO]{\thepage}

\renewcommand{\headrulewidth}{2pt}
\renewcommand{\footrulewidth}{1pt}
\renewcommand{\headrule}{\hbox to\headwidth{\color{lime}\leaders\hrule height \headrulewidth\hfill}}
\renewcommand{\footrule}{\hbox to\headwidth{\color{lime}\leaders\hrule height \footrulewidth\hfill}}

\hypersetup{
  pdftitle={Smart Agriculture},
  pdfauthor={Roberto Chiper},
  pdfkeywords={agriculture, AI, smart farming, CNN, XGBoost},
  bookmarksnumbered,
  pdfstartview={FitH},
  urlcolor=cyan,
  colorlinks=true,
  linkcolor=red,
  citecolor=green,
}

\setcounter{secnumdepth}{3}
\setcounter{tocdepth}{3}

\linespread{1.5}
\justifying

\makeindex

\begin{document}

\begin{titlepage}
\sloppy
\begin{center}
BABE\c{S} BOLYAI UNIVERSITY, CLUJ NAPOCA, ROM\^ANIA

FACULTY OF MATHEMATICS AND COMPUTER SCIENCE

\vspace{6cm}

\Huge \textbf{Smart Agriculture}

\vspace{1cm}

\normalsize -- Bachelor Thesis --
\end{center}

\vspace{5cm}

\begin{flushright}
\Large{\textbf{Student:}} \\
Roberto-Marian Chiper, Informatic\u{a}, anul III, roberto@email.com
\end{flushright}

\vspace{4cm}

\begin{center}
2024-2025
\end{center}
\end{titlepage}

\pagenumbering{gobble}

\begin{abstract}
This thesis presents an intelligent system to support decision-making in agriculture.
\begin{itemize}
  \item The proposed system addresses crop recommendation, disease detection, and yield estimation.
  \item Artificial Intelligence (AI) models such as eXtreme Gradient Boosting (XGBoost) and Convolutional Neural Networks (CNN) are employed.
  \item Public and real-world datasets are used for model training and evaluation.
  \item Results demonstrate high accuracy and strong potential for scalability.
\end{itemize}

\noindent\textbf{Author:} Roberto-Marian Chiper \\
\textbf{Date:} 2024–2025
\end{abstract}

\tableofcontents
\listoftables
\listoffigures
\listofalgorithms

\newpage
\pagenumbering{arabic}

\chapter{Introducere}
Această lucrare se încadrează în domeniul agriculturii inteligente, un subdomeniu emergent care utilizează tehnologii moderne pentru a spori eficiența și sustenabilitatea agriculturii. În acest capitol este oferit contextul problemei abordate, utilitatea acesteia, precum și obiectivele și structura lucrării.

\section{Contextul actual al agriculturii inteligente}
În contextul provocărilor globale legate de schimbările climatice, creșterea populației și necesitatea sustenabilității, sectorul agricol este forțat să evolueze rapid. Agricultura tradițională nu mai este suficientă pentru a răspunde cerințelor moderne privind productivitatea, eficiența și protejarea mediului. În acest peisaj, \textbf{agricultura inteligentă} (Smart Agriculture) apare ca o soluție promițătoare, integrând tehnologii moderne precum \textit{inteligența artificială (AI)}, \textit{internetul lucrurilor (IoT)}, \textit{senzorii} și \textit{sistemele automate} pentru a optimiza procesele agricole.

Această abordare tehnologică permite colectarea și analiza unor volume mari de date, ceea ce duce la decizii mai informate și adaptate condițiilor locale. Astfel, fermierii pot beneficia de predicții precise, monitorizare în timp real și intervenții automatizate pentru a îmbunătăți randamentul și calitatea producției.

\section{Scopul și obiectivele lucrării}
Scopul acestei lucrări de licență este de a proiecta și implementa un sistem inteligent care asistă procesul decizional în agricultură prin metode de învățare automată. Lucrarea propune o aplicație practică, organizată în trei funcționalități principale (iterații):

\begin{itemize}
  \item \textbf{Iterația 1}: Recomandarea unei culturi potrivite în funcție de tipul de sol și alți parametri.
  \item \textbf{Iterația 2}: Detecția bolilor în frunzele plantelor, pe baza unor imagini analizate de un model de tip CNN.
  \item \textbf{Iterația 3}: Estimarea randamentului (yield) unei culturi, pe baza unor date concrete despre sol, plantă și climat.
\end{itemize}

Obiectivele lucrării sunt:
\begin{itemize}
  \item identificarea și prelucrarea unor seturi de date relevante pentru fiecare funcționalitate;
  \item alegerea și testarea unor algoritmi de inteligență artificială potriviți (ex: XGBoost, CNN);
  \item dezvoltarea unei aplicații interactive și ușor de utilizat;
  \item validarea rezultatelor obținute prin metode specifice de evaluare și testare.
\end{itemize}

\section{Structura lucrării}
Lucrarea este structurată în șapte capitole, după cum urmează:

\begin{itemize}
  \item \textbf{Capitolul 2} oferă o prezentare generală a domeniului agriculturii inteligente, cu accent pe tehnologii, provocări și rolul datelor.
  \item \textbf{Capitolul 3} discută aplicațiile inteligenței artificiale în agricultură și algoritmii utilizați frecvent.
  \item \textbf{Capitolul 4} descrie problema abordată, datele folosite, metodele de preprocesare, antrenare, evaluare și modelele aplicate.
  \item \textbf{Capitolul 5} este dedicat aplicației inteligente dezvoltate: arhitectura, scenarii de utilizare, tehnologii și testarea codului.
  \item \textbf{Capitolul 6} sintetizează principalele concluzii, identifică limitările soluției propuse și direcțiile viitoare.
  \item În final, este inclusă o \textbf{analiză SWOT} și o scurtă discuție despre \textbf{etica} soluției dezvoltate.
\end{itemize}

\chapter{Agricultură Inteligentă – Concept și Tehnologii}

\section{Definirea conceptului de Smart Agriculture}
Agricultura inteligentă reprezintă o abordare modernă a activităților agricole, care combină tehnologia digitală cu analiza avansată a datelor pentru a îmbunătăți eficiența și sustenabilitatea fermelor. Prin integrarea unor soluții precum senzori, sisteme GPS, drone, platforme software și algoritmi de inteligență artificială, agricultura devine un proces ghidat de informații în timp real, nu doar de experiență empirică.

Scopul principal al agriculturii inteligente este de a asigura o utilizare eficientă a resurselor (apă, fertilizanți, teren), reducând în același timp riscurile legate de variațiile climatice, boli sau dăunători. Această tranziție către agricultură de precizie este sprijinită de avansurile recente în IoT și AI, cu impact dovedit în planificarea agricolă și optimizarea deciziilor strategice~\cite{wolfert2017big,liakos2018machine}.

% --ADDED TEXT--
În ultimii ani, au început să se contureze perspective noi în care infrastructura de comunicații joacă un rol critic în susținerea agriculturii inteligente. Tehnologiile 5G și viitorul standard 6G promit viteze mult mai mari de transmitere a datelor și latențe reduse, facilitând o conectivitate extinsă între senzori, echipamente și ferme îndepărtate~\cite{zhang2022_6g_agriculture}. Astfel de progrese pot contribui la apariția unui ecosistem ultra-conectat, în care fermierii pot reacționa în timp real la schimbările de mediu și pot implementa soluții predictive avansate.

Totodată, literatura arată că abordările bazate pe IoT se află la baza multor soluții de monitorizare și control în agricultură, oferind date detaliate și actualizate despre cultură, condiții climatice și starea solului~\cite{navarro2020_iot_smartfarming}. Această bază solidă de informații, colectată de dispozitive inteligente, permite optimizarea resurselor și reducerea intervențiilor manuale, contribuind la dezvoltarea unei agriculturi mai eficiente și sustenabile.

\section{Provocări actuale în domeniu}

Implementarea agriculturii inteligente aduce numeroase beneficii, însă se confruntă și cu provocări semnificative care pot împiedica adoptarea pe scară largă a acestor tehnologii.

\paragraph{Costurile inițiale ridicate} Investițiile necesare pentru achiziționarea și instalarea echipamentelor inteligente, precum senzori, drone sau sisteme de analiză a datelor, pot fi prohibitive pentru micii fermieri. Acest aspect limitează accesul la tehnologiile moderne și poate accentua disparitățile între fermele mari și cele mici.

\paragraph{Conectivitatea redusă în zonele rurale} Multe regiuni agricole se confruntă cu o infrastructură de internet slabă sau inexistentă, ceea ce afectează negativ funcționarea sistemelor bazate pe cloud sau IoT. Lipsa unei conexiuni stabile împiedică transmiterea și analiza datelor în timp real, esențială pentru luarea deciziilor informate.

\paragraph{Lipsa competențelor tehnice} Mulți fermieri nu dispun de cunoștințele necesare pentru a opera și întreține sistemele avansate de agricultură inteligentă. Această lipsă de expertiză poate duce la utilizarea ineficientă a tehnologiilor sau la evitarea completă a acestora.

\paragraph{Integrarea dificilă a datelor} Colectarea datelor din surse multiple, precum senzori, imagini satelitare sau rapoarte meteorologice, necesită standarde comune și platforme compatibile. În absența acestora, integrarea și analiza datelor devin complicate și consumatoare de timp.

\paragraph{Riscuri legate de confidențialitatea datelor} Utilizarea pe scară largă a tehnologiilor digitale în agricultură ridică probleme legate de securitatea și confidențialitatea datelor. Fermierii pot fi reticenți în a partaja informații sensibile despre practicile lor agricole, temându-se de posibile abuzuri sau concurență neloială.

Aceste provocări sunt recunoscute la nivel global și reprezintă obstacole majore în calea adoptării pe scară largă a agriculturii inteligente, în special în țările în curs de dezvoltare~\cite{wolfert2017big,kamilaris2017bigdata}.

Conform literaturii recente, aceste provocări sunt comune la nivel global și reprezintă un obstacol major în scalarea soluțiilor de tip smart farming, mai ales în țările în curs de dezvoltare~\cite{wolfert2017big,kamilaris2017bigdata}.

% --ADDED TEXT--
De asemenea, cercetările recente subliniază că inadecvarea infrastructurii și costurile ridicate de implementare, mai ales în contextul rețelelor complexe de date și echipamente, conduc la o rată de adopție scăzută chiar și în țările cu o economie în plină dezvoltare~\cite{ahmed2024_smart_agriculture}. În plus, lipsa competențelor digitale și a unor strategii guvernamentale clare poate amplifica dificultățile fermierilor în a utiliza tehnologii avansate, evidențiind necesitatea unor programe de formare și politici de susținere dedicate.

Un alt aspect esențial este reticența fermierilor față de gestionarea și partajarea datelor, pe fondul absenței unor standarde bine definite și a unor garanții solide de securitate~\cite{talero2023_smartfarming_barriers}. Acest fenomen, combătut cu dificultate în țările emergente, este agravat de costurile ridicate și de lipsa unui cadru normativ favorabil adoptării Agriculturii 4.0~\cite{islam2023_agriculture4_emerging}. În ciuda acestor obstacole, se remarcă o tendință generală de creștere a interesului pentru digitalizarea agriculturii, indicând o nevoie urgentă de armonizare între factorii economici, tehnologici și legislația în domeniu.

\section{Rolul datelor și senzorilor în agricultură}
Datele reprezintă fundamentul oricărui sistem agricol inteligent. Informațiile colectate de senzori și echipamente de monitorizare permit:

\begin{itemize}
  \item Observarea continuă a stării solului și a culturilor;
  \item Optimizarea resurselor (apă, fertilizanți, pesticide);
  \item Detectarea rapidă a anomaliilor;
  \item Estimarea randamentului și luarea deciziilor bazate pe analize predictive.
\end{itemize}

Senzorii (instalați în sol, pe plante sau integrați în drone) colectează date precum temperatura, umiditatea, pH-ul, nivelul de azot sau intensitatea luminii. Aceste date sunt centralizate și analizate automat de sisteme software, care pot oferi sugestii sau intervenții automatizate~\cite{liakos2018machine}.

Imagistica oferită de drone sau sateliți, alături de analize multispectrale și tehnici de machine learning, contribuie semnificativ la identificarea bolilor, estimarea randamentului și cartografierea terenurilor~\cite{zhang2002precision}.

% --ADDED TEXT--
În contextul agriculturii de precizie, senzorii inteligenți şi rețelele wireless au devenit elemente cheie, fiind esențiale pentru o supraveghere continuă și pentru colectarea datelor la nivel micro, în fiecare zonă a terenului agricol~\cite{soussi2024_smartsensors}. Această abordare granulară permite intervenții adaptate specificului fiecărui lot, maximizând utilizarea resurselor și reducând pierderile cauzate de dăunători sau condiții nefavorabile.

Pe măsură ce volumele de date cresc, instrumentele de tip big data și analytics devin indispensabile în procesarea și interpretarea informațiilor culese de senzori, facilitând recomandări rapide și algoritmi predictivi avansați~\cite{alahmad2023_iot_bigdata}. Astfel, rolul datelor nu se limitează doar la monitorizare, ci capătă o componentă strategică, prin care agricultorii își pot planifica resursele, pot anticipa schimbările climatice și pot lua decizii operative mai informate.


\chapter{Inteligența Artificială în Agricultură}
\section{Aplicații AI: clasificare, predicție, optimizare}
\section{Algoritmi utilizați frecvent (XGBoost, CNN)}
\section{Avantaje și limitări}

\chapter{Date și Modele AI Utilizate}
\section{Specificarea problemei: ce se dă și ce se cere}
\section{Seturi de date utilizate}
\section{Preprocesare și curățare}
\section{Explorare vizuală și observații asupra distribuțiilor}
\section{Algoritmi utilizați (XGBoost, CNN)}
\section{Antrenarea modelelor}
\section{Evaluare și rezultate (matrice de confuzie, F1-score, grafice comparative)}
\section{Strategii de optimizare (early stopping, tuning)}
\section{Alegerea modelului final și integrarea în aplicație}

\chapter{Sistemul Inteligent Dezvoltat}
\section{Arhitectura soluției propuse}
\section{Fluxul de date în aplicație}
\section{Scenarii de utilizare}
\section{Limbaje, framework-uri și tehnologii folosite}
\section{Testarea și validarea codului (unit testing, asserts, coverage)}

\chapter{Concluzii, Analiză SWOT și Etică}
\section{Observații asupra performanței modelelor}
\section{Limitările soluției actuale}
\section{Direcții viitoare}
\section{Analiza SWOT}
\section{Considerații etice (bias, transparență, securitate, responsabilitate)}

\bibliographystyle{plain}
\bibliography{references} 

\end{document}
